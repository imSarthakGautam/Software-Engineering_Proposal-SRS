\section{FEASIBILITY STUDY}
\subsection{Economic Feasibility}
Given that the proposed "Life Tracker" system is based on a web application utilizing the MERN stack, we'll leverage the advantages of these widely used open-source technologies. The MERN stack offers a seamless and cost-effective approach to building a dynamic and responsive application, ensuring a high-quality user experience while minimizing development expenses.The primary components of the MERN stack are well-established and supported by a large developer community. The use of open-source tools eliminates substantial licensing costs, making our project economically viable.
\subsection{Operational Feasibility}
Operational feasibility for the "Life Tracker" web app is driven by key factors like user-friendliness, minimal training needs, and manageable support requirements. Its UI is user friendly. The app's interface is designed to be intuitive, ensuring users can easily navigate without advanced technical knowledge. Maintenance and support are expected to be straightforward, reducing the need for complex technical assistance. Overall, the "Life Tracker" web app is operationally feasible, offering a streamlined experience for users while keeping support efforts manageable.

\subsection{Technical Feasibility}
The "Life Tracker" web app utilizes the MERN (MongoDB, Express.js, React, Node.js) stack for technical feasibility. This stack incorporates widely adopted open-source technologies with strong community support. React and Node.js are used for frontend and backend development, respectively. These technologies are well-established with substantial backing from reputable organizations which ensures the availability of extensive technical support within the development community. Therefore, the "Life Tracker" project is technically feasible with rich well-supported technologies.




